\documentclass[10pt, letterpaper]{article}

% Packages:
\usepackage[
    ignoreheadfoot, % set margins without considering header and footer
    top=2 cm, % seperation between body and page edge from the top
    bottom=1 cm, % seperation between body and page edge from the bottom
    left=2 cm, % seperation between body and page edge from the left
    right=2 cm, % seperation between body and page edge from the right
    footskip=1.0 cm, % seperation between body and footer
    % showframe % for debugging 
]{geometry} % for adjusting page geometry
\usepackage{titlesec} % for customizing section titles
\usepackage{tabularx} % for making tables with fixed width columns
\usepackage{array} % tabularx requires this
\usepackage[dvipsnames]{xcolor} % for coloring text
\definecolor{primaryColor}{RGB}{0, 0, 0} % define primary color
\usepackage{enumitem} % for customizing lists
\usepackage{fontawesome5} % for using icons
\usepackage{amsmath} % for math
\usepackage[
    pdftitle={John Doe's CV},
    pdfauthor={John Doe},
    pdfcreator={LaTeX with RenderCV},
    colorlinks=true,
    urlcolor=primaryColor
]{hyperref} % for links, metadata and bookmarks
\usepackage[pscoord]{eso-pic} % for floating text on the page
\usepackage{calc} % for calculating lengths
\usepackage{bookmark} % for bookmarks
\usepackage{lastpage} % for getting the total number of pages
\usepackage{changepage} % for one column entries (adjustwidth environment)
\usepackage{paracol} % for two and three column entries
\usepackage{ifthen} % for conditional statements
\usepackage{needspace} % for avoiding page brake right after the section title
\usepackage{iftex} % check if engine is pdflatex, xetex or luatex

% Ensure that generate pdf is machine readable/ATS parsable:
\ifPDFTeX
    \input{glyphtounicode}
    \pdfgentounicode=1
    \usepackage[T1]{fontenc}
    \usepackage[utf8]{inputenc}
    \usepackage{lmodern}
\fi

\usepackage{charter}

% Some settings:
\raggedright
\AtBeginEnvironment{adjustwidth}{\partopsep0pt} % remove space before adjustwidth environment
\pagestyle{empty} % no header or footer
\setcounter{secnumdepth}{0} % no section numbering
\setlength{\parindent}{0pt} % no indentation
\setlength{\topskip}{0pt} % no top skip
\setlength{\columnsep}{0.15cm} % set column seperation
\pagenumbering{gobble} % no page numbering

\titleformat{\section}{\needspace{4\baselineskip}\bfseries\large}{}{0pt}{}[\vspace{1pt}\titlerule]

\titlespacing{\section}{
    % left space:
    -1pt
}{
    % top space:
    0.3 cm
}{
    % bottom space:
    0.2 cm
} % section title spacing

\renewcommand\labelitemi{$\vcenter{\hbox{\small$\bullet$}}$} % custom bullet points
\newenvironment{highlights}{
    \begin{itemize}[
        topsep=0.10 cm,
        parsep=0.10 cm,
        partopsep=0pt,
        itemsep=0pt,
        leftmargin=0 cm + 10pt
    ]
}{
    \end{itemize}
} % new environment for highlights


\newenvironment{highlightsforbulletentries}{
    \begin{itemize}[
        topsep=0.10 cm,
        parsep=0.10 cm,
        partopsep=0pt,
        itemsep=0pt,
        leftmargin=10pt
    ]
}{
    \end{itemize}
} % new environment for highlights for bullet entries

\newenvironment{onecolentry}{
    \begin{adjustwidth}{
        0 cm + 0.00001 cm
    }{
        0 cm + 0.00001 cm
    }
}{
    \end{adjustwidth}
} % new environment for one column entries

\newenvironment{twocolentry}[2][]{
    \onecolentry
    \def\secondColumn{#2}
    \setcolumnwidth{\fill, 4.5 cm}
    \begin{paracol}{2}
}{
    \switchcolumn \raggedleft \secondColumn
    \end{paracol}
    \endonecolentry
} % new environment for two column entries

\newenvironment{threecolentry}[3][]{
    \onecolentry
    \def\thirdColumn{#3}
    \setcolumnwidth{, \fill, 4.5 cm}
    \begin{paracol}{3}
    {\raggedright #2} \switchcolumn
}{
    \switchcolumn \raggedleft \thirdColumn
    \end{paracol}
    \endonecolentry
} % new environment for three column entries

\newenvironment{header}{
    \setlength{\topsep}{0pt}\par\kern\topsep\centering\linespread{1.5}
}{
    \par\kern\topsep
} % new environment for the header

\newcommand{\placelastupdatedtext}{% \placetextbox{<horizontal pos>}{<vertical pos>}{<stuff>}
  \AddToShipoutPictureFG*{% Add <stuff> to current page foreground
    \put(
        \LenToUnit{\paperwidth-2 cm-0 cm+0.05cm},
        \LenToUnit{\paperheight-1.0 cm}
    ){\vtop{{\null}\makebox[0pt][c]{
        \small\color{gray}\textit{Last updated in July 2024}\hspace{\widthof{Last updated in July 2024}}
    }}}%
  }%
}%

% save the original href command in a new command:
\let\hrefWithoutArrow\href

% new command for external links:


\begin{document}
    \newcommand{\AND}{\unskip
        \cleaders\copy\ANDbox\hskip\wd\ANDbox
        \ignorespaces
    }
    \newsavebox\ANDbox
    \sbox\ANDbox{$|$}

    \begin{header}
        \fontsize{25 pt}{25 pt}\selectfont Hannah Cutler

        \vspace{5 pt}

        \normalsize
        \mbox{Santa Barbara, CA}%
        \kern 5.0 pt%
        \AND%
        \kern 5.0 pt%
        \mbox{\hrefWithoutArrow{mailto:hannahcutler@ucsb.edu}{hannahcutler@ucsb.edu}}%
        \kern 5.0 pt%
        \AND%
        \kern 5.0 pt%
        \mbox{\hrefWithoutArrow{tel:(206) 488-8441}{(206) 488-8441}}%
        \kern 5.0 pt%
        \AND%
        \kern 5.0 pt%
        \mbox{\hrefWithoutArrow{www.linkedin.com/in/hannahcutler2005}{www.linkedin.com/in/hannahcutler2005}}%
        \kern 5.0 pt%
        \AND%
        \kern 5.0 pt%
    \end{header}

    \vspace{5 pt - 0.3 cm}

    \section{Education}



        
        \begin{twocolentry}{
            \item  Santa Barbara, CA
            \item \textit{Expected June 2027}
        }
            \item \textbf{UC Santa Barbara}
            \item \textit{Bachelot of Science (B.S.), Electrical Engineering}
            \end{twocolentry}
            

        \vspace{0.10 cm}



    
    \section{Work Experience}


    \begin{twocolentry}{
        June 2024 – Present
    }
        \textbf{REU Researcher}, University of California, San Diego -- San Diego, CA\end{twocolentry}

    \vspace{0.10 cm}
    \begin{onecolentry}
        \begin{highlights}
            \item Selected as one of 10 participants from across the country to conduct engineering research in collaboration with Scripps Institute of Oceanography
            \item Working in a 4-person team to develop the 3rd generation Smartfin, to collect and transmit oceanographic data in the costal surf-zone.
            \item Implementing Python simulations to understand the effects of spectral artifacts on 3 different sensors
        \end{highlights}
    \end{onecolentry}


    \vspace{0.2 cm}
    
    \begin{twocolentry}{
            Jan 2024 – Present
        }
            \textbf{Hardware Development Intern}, Hikari Medical Technologies -- Santa Barbara, CA\end{twocolentry}

        \vspace{0.10 cm}
        \begin{onecolentry}
            \begin{highlights}
                \item Utilizing iterative design process to design and improve a functional and wearable housing for the device, to align lens, laser diode, and sensor (AS7263) while keeping device dimensions under 50mm x 40 mm.
                \item Recommending, designing, and integrating PCB with 3D printed case. Test each design 1-2 times using accurate conditions to ensure functionality.
                \item Communicating regularly with company co-founders to align design goals and update the design process.
            \end{highlights}
        \end{onecolentry}

        \vspace{0.2 cm}

        \begin{twocolentry}{
            Oct 2023 – Present
        }
            \textbf{Controls Hardware Member}, Formula SAE -- Santa Barbara, CA\end{twocolentry}

        \vspace{0.10 cm}
        \begin{onecolentry}
            \begin{highlights}
                \item Utilizing wiring harness software RapidHarness to collaboratively design and manufacture a functional and cost-efficient wiring harness for final car in a team of 3.
                \item Translating wiring harness diagrams from software to ordering list, utilizing CAD models to ensure correct wire length measurements. 
                \item Leveraging Fusion360 to design printed circuit boards (PCBs), combining 2-3 sensors into a board. Maintain PCB design best practices to uphold uniformity with Gaucho Racing designs.
            \end{highlights}
        \end{onecolentry}


        \vspace{0.2 cm}

    \section{Leadership Experience}

        \begin{twocolentry}{
            May 2024 – Present
        }
            \textbf{External Vice President}, SWE-UCSB -- Santa Barbara, CA\end{twocolentry}

        \vspace{0.10 cm}
        \begin{onecolentry}
            \begin{highlights}
                \item Contacting and maintaining relationships with 90+ companies to support club finances and networking opportunities 
                \item Proposing and instituting new initiatives such as SWE Tech Team @ UCSB that work to close the gender gap in engineering graduates.
                \item Organizing annual Evening With Industry event with 75+ participants, up to 20 sponsors, and thousands of dollars of investment.
            \end{highlights}
        \end{onecolentry}

    
    \section{Projects}



        
        \begin{twocolentry}{
    
        }
            \textbf{3-Dimensional Virtual Keyboard}\end{twocolentry}

        \vspace{0.10 cm}
        \begin{onecolentry}
            \begin{highlights}
                \item Proposed and cooperatively designed gloves with 2 integrated GPU-6050's and 5 flex sensors to calculate data feedback and play
                108 corresponding keyboard notes using an Arduino-based microcontroller
                \item Arranged prototype breadboards to facilitate easy debugging. Researched, wrote, and debugged Arduino code in C++ over the
                course of 5 in person lab meetings.
            \end{highlights}
        \end{onecolentry}
    
    \section{Technical Skills}

        \begin{onecolentry}
            \item Python, C++, Java, Fusion360, AutoCAD, Altium, through hole and SMD soldering.
        \end{onecolentry}

\end{document}
